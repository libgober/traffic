\section{Conclusion}
This study has found that the introduction of E-ZPass brought about a significant rise in property values, and was also associated with a significant decrease in various measures of support for Democratic Presidential candidates. These relationships were estimated using conditional diff-in-diff and instrumental variables, techniques not frequently used in the study of political behavior. We have argued that this relationship between property values and support for Democratic candidates is causal, since EZ-Pass has not altered the environment in other ways that are significant enough to explain our results. 

Although our study has only looked at the introduction of EZ-Pass in the Eastern United States, the design is easily transported to other geographical contexts, both within the US and abroad. One interesting question for future research is whether the strength of the property-voting relationship depends on a community's wealth level. We have no real evidence to answer this question, but by analogy to income \parencite{Gelman2007}, we might expect the effect to be stronger in political communities that are poorer. Another important question is \emph{why} property values cause this effect. \textcite{Ansell2014} argues that a rise in property values gives individuals more ``permanent  income'' that can be converted via sale or borrowing, decreasing demand for social insurance. Others might argue that the effect is mostly due to the increased salience of property taxes. One could imagine a design that sought to sort out these explanations by using survey data rather than vote outcomes. The primary challenge for such a study is finding individuals who live close enough to a highway and therefore can be considered as participating in the study; if it were easy to find enough individuals using existing survey data, we would have done so. Yet with advance planning, such a study clearly could be done, since govermental entities typically announce EZ-Pass well in advance of installation. Such a panel survey might give political science scholarship additional causal leverage on the questions of how individual economic circumstances affect their attitudes. 
 
More broadly, we see our work as helping build an explanation for political polarization in the United States based on patterns of internal migration. Internal migration increases property values in states that gain citizens and decreases property values in states that lose them. The migrants themselves experience little change in individual wealth, at least in the short-term. However, those who already own homes in net recipient states should receive a windfall in their individual wealth. In the United States, migrants are predominantly leaving blue states and going to red states. Thus, internal migration has the effect of gradually making red states purple via a composition change, while pushing the already conservative home owners in red states even further to the right. More works needs to be done to evaluate such a theory, which we can only sketch in brief here. 

In sum, this work has used a novel identification strategy to address an important question at the intersection of political behavior and political economy. It suggests that scholars and political observers should consider citizens' entire economic portfolio, and not just income, in analyzing American politics. \hfill $\square$ 