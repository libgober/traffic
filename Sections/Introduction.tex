\section{Introduction}
This study uses the introduction of E-ZPass in order to identify the effect of changes in property values on support for US political parties. In 2002, E-ZPass was introduced along toll roads in New Jersey and Pennsylvania. Drivers who purchased E-ZPass transponders were able to avoid manual toll collection, while driver who did not purchase the transponder nonetheless benefited from shorter lines at the manual tolls. According to estimates published by the Department of Transportation of New Jersey, the introduction of E-ZPass reduced total delays at toll locations 85\% in the year after its adoption, resulting in an average decrease of about 10\% in the daily commute of those who used highways with E-ZPasses installed \parencite{NJT2001}. We would expect that, over time, driving behaviors would change as more individuals elect to use the now speedier toll roads. This change might have the effect of reducing traffic on local streets as well. Communities near some sections of highway saw no substantial change in traffic patterns, whereas communities near other sections of highway saw substantial decreases in travel times and less congestion on local roads. As a consequences, we would expect (and find) that communities near newly introduced E-ZPasses receive a positive wealth shock in the form of higher property values. The introduction of E-ZPass is thus fairly unique in that it creates intra-state variation in property values, but without creating the side-effects typically associated with government stimulus programs. We explore the effect of changes in property values on how citizens express their support for political parties at the national level.

Although land values have been widely-recognized as an important variable in macro-economic models and by historical political economy, the topic remains underexplored by scholars of contemporary political behavior. This is surpsing. Most Americans today have invested a plurality of their wealth in their homes, with mortgages also constituting the single biggest debt obligation that households face. According to most recently available statistics, the US homeownership rate stood at about 65 percent. Compared to renters, homeowners are more likely to vote, to be conservative, and to hold strong policy preferences (Gilderbloom and Markham, 1995; Kingston, Thompson, Eichar, 1984; Schwartz, 2008). As Lacy and Soskice (2015) write, ``the decisive voter in local elections is likely to be a home owner.'' This group thus constitutes a pivotal constituency in American politics. 

One reason that scholars of American politics have not paid sufficient attention to residential property is federalism. In the American system many of the most important decisions facing homeowners - such as zoning, schooling, and policing - are made primarily at the state or local level. Yet many aspects of federal policy do affect home values. Indeed, home owners are directly impacted by federal tax and monetary policy, to say nothing of the myriad federal programs designed to support home ownership, or the bank bailout of 2008. Although historically there has been bipartisan support for programs increasing homeownership, the fact that both parties have supported similar polices does not mean that such policies have benefited each party equally.

Methodologically, we isolate causal effects using a conditional difference-in-difference estimator, using matching as well as bootstrap and block bootstrap standard errors. Instrumental variable (IV) estimates are also presented. Throughout the study, we primarily examine precinct-level election outcomes from 2000 and 2004. In both elections, property and taxation were at the center of the political debate, providing an ideal setting for this analysis. We also provide a detailed portrait of the communities impacted by E-ZPass to confirm that E-ZPass was associated with no other significant socio-economic changes. We areal interpolate census block-group data to provide state-of-the-art estimates of the change in racial and economic makeup of communities near the infrastructure improvement. We find that the introduction of E-ZPass did not significantly impact the socio-economic attributes of citizens. E-ZPass is primarily associated with a steep rise in property values, and also with an increase in conservative political behavior. These empirical results are consistent with a formal model developed in the Appendix describing the relationship between property ownership and voting behaviors. By conducting a sensitivity analysis, placebo test, and using multiple dependent variables, we demonstrate that our results are robust.