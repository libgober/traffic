\begin{abstract}
\noindent This study uses the introduction of E-ZPass in Pennsylvania and New Jersey in order to identify the effect of changing property values on support for US political parties. We show that E-ZPass caused a decline in support for Democratic Presidential candidates. We argue that this effect is not spurious, but rather due to the fact that precincts near E-ZPass toll locations experienced a downward shock to transportation cost and a corresponding rise in average home price. We rule out other explanations dealing with turnout, community change, or rising incomes, and replicate the design in Ohio. By conducting a sensitivity analysis, placebo test, and using multiple dependent variables, we demonstrate robustness. 
\end{abstract} 
%note: the abstract should be 150 words or less. 