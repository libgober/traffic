\begin{abstract}
\noindent A rich literature exists on the extent to which homeownership is central to American political attitudes. This paper uses the introduction of E-ZPass in Pennsylvania and New Jersey to identify the effect of traffic-reducing transportation infrastructure on property values and, in turn, political behaviors. We develop a model showing that faster travel times results in individuals preferring a lower tax rate, as those who face the lower travel times are made effectively wealthier. Next, we present empirical evidence consistent with this theoretical result. We show that voting precincts near newly introduced E-ZPass toll plazas experienced a sharp increase in property values relative to similar precincts near non-E-ZPass exists, giving us leverage to identify the causal effect of property value changes on voting. We find that positive shocks in property values are associated with an increase in Republican vote share. 
\end{abstract} 