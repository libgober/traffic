\section{Relationship to the Literature}
At least since the 1940s, social scientists have relied on survey research to understand the factors affecting voting behavior, but only recently have begun to use experimental and observational evidence to gain ``causal leverage for analyses of voting behavior'' \parencite{Bartels2010}. Early studies of voting behavior established that individual vote choice is susceptible to election-year specific shocks, such as an unusually popular presidential candidate. However, these studies also found that the way an individual votes in most elections is largely a function of their personal identity characteristics \parencite{Converse1966}. According to this line of research, the most significant determinants of long-term voting behavior include party identification, ethnicity, gender, age, religion, education, and occupation \parencite{Lazarsfeld1948,Berelson1954,Campbell1960,Stanley2006a}. Noteworthy too are the factors these studies did \textit{not} find significant for long-term voting behavior: preferences about political issues, for example, and economic self-interest. This is not to say that financial well-being was ever considered irrelevant for voting behavior; the literature is replete with studies demonstrating that the success of incumbents is tied to that of the economy \parencite{Tufte1975,Meltzer1975,Hibbs1987}. Yet it has been argued that this effect is largely based on ``sociotropic'' evaluations of the national economy, not personal economic experience. Evidence for the latter's effect on voting behavior has been mixed \parencite{Linn2010}. In this way, early survey research argued that long-term voting behavior was mostly reducible to an individual's demographic characteristics, and that, all else equal, increasing an individual's income would not change their vote choice. This view is in considerable tension with formal models whose comparative statics expect such a relationship between an indivdual's economic situation and their voting behavior \parencite[e.g.,][]{Romer1975,Roberts1977,Meltzer1981}.

Although the conclusions of survey research just described have been replicated numerous times \parencite{Nie1976,Smith989,LewisBeck2008}, in the past decade, these results have faced significant challenges from scholars bringing to bear new data sources and applying new statistical techniques. In particular, ideology and economic interest are gaining renewed recognition as determinants of life-long voting behavior. \textcite{Ansolabehere2008} show that the apparent incoherence of individual issue attitudes may largely be a result of measurement error. If one averages multiple question responses, estimates of individual issue attitudes are stabilized, and become nearly as predictive of voting behavior as partisan identification. \textcite{Gelman2007} use multilevel-modeling of exit poll data to establish that income is an important predictor of individual vote behavior, but its predictive power depends greatly on geography: in poor states like Alabama, the poor vote very differently from wealthier citizens, while in rich states like Connecticut the difference is barely present. \textcite{Hersh2015}, building on the results of \textcite{Gelman2007}, use highly disaggregated registration, census, and election returns to show that income significantly influences voting behavior only in Congressional districts with large minority populations. Lastly, \citeauthor{Arunchalam} use height as an instrument for studying the effect of income on support for the Conservative Party in the United Kingdom. 

This article continues in the vein of these most recent papers, using data sources and methods not frequently used in the voting behavior literature to support the claim that economic experience can have a significant effect on vote choice. Our decision to examine property values rather than income naturally invites comparison with the approach of \textcite{Ansell2014}. In his paper, Ansell subsets ANES survey data to Metropolitain Statistical Areas where the Federal Housing Adminstration (FHA) collects data on home values. He uses these data to calculate the effect of home value appreciation on attitudes toward a redistributive program. He finds that increasing property values makes individuals less likely to support redistribution. For robustness, he looks at similar data from the UK as well as cross-country survey data. Ansell's paper nicely compliments ours. His data sources do not readily lend themselves to causal identification strategies, whereas ours do, while our outcome variables of donations and votes are only rough proxies for political attitudes, which are better captured by surveys. Together, these papers present consistent evidence of a substantial connection that increasing property values make individuals less supportive of redistribution and more likely to support conservative political parties. 

Our work also implicitly relies on insights developed in the study of urban economics and economic geography. The ``monocentric city model'' has a long history within economics and seeks to explain how the economy organizes space. First proposed by \textcite{Thunen1826} to study crop usage, this model and its refinements remain widely used in the theoretical and empirical literature \parencite{Alonso,Fujita1999,BaumSnow2007}. For our purposes, an important implication of such models is that the price individuals are willing to pay for residential property is inversely related to the costs of getting from that property to the city center. Hence, a change in travel costs should result in a change in property values. The idea of exploiting the introduction of E-ZPass for econometric inference comes from \textcite{Currie2011a}, who look at the effect of decreased traffic and car polluton on infant health. Considerable evidence supports the conclusion that residential property values are responsive to shocks that affect travel costs \parencite{Levkovich2016}, and indeed we confirm that this is true for the E-ZPass introduction. Changes in property values may eventually result in changes to who is able to live in communities, but in the short term, residents will not necessarily be willing or able to sell their homes. In the Appendix, we formalize these notions by developing a two period model: in the first period, individuals choose homes given a personal endowment and assuming they will have to pass a balanced budget to support the provision of public goods. In the second, they either receive a reducation in travel costs or do not. In line with our informal predictions, the formal model produces comparative statics showing that those who receive a windfall reduction in travel costs are more reluctant to support redistributive programs. 

