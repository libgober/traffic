\section{Relationship to the Literature}
At least since the 1940s, social scientists have relied on survey research to understand the factors affecting voting behavior, yet only recently have scholars begun to use experimental and observational evidence to gain ``causal leverage for analyses of voting behavior" \parencite{Bartels2010}. Early studies of voting behavior established that although individual vote choice is susceptible to election-year specific shocks, for example an unusually popular presidential candidate or war-time discontent, the way an individual votes in most elections is for the most part a function of their personal identity characteristics \parencite{Converse1966}. The most significant determinants of long-term voting behavior according to this line of research include party identification, ethnicity, gender, age, religion, education, and occupation \parencite{Lazarsfeld1948,Berelson1954,Campbell1960,Stanley2006a}. Noteworthy too are the factors these studies did \textit{not} find significant for long-term voting behavior: preferences about political issues, for example, and economic self-interest. This is not to say that financial well-being was ever considered irrelevant for voting behavior; the literature is replete with studies demonstrating that the success of incumbents is tied to that of the economy \parencite{Tufte1975,Meltzer1975,Hibbs1987}. Yet this effect appears to be based largely on "sociotropic" evaluations of national economic performance, not personal economic experience, and evidence for the latter's effect on voting behavior is at best mixed, even in the short term \parencite{Linn2010}.  For early survey research, long-term voting behavior was for the most part reducible to an individual's demographic characteristics, and would not predict that increasing an individual's wealth or income would ipso facto change their vote choice.  This view is in considerable tension with formal models whose comparative statics expect such a relationship between an indivdual's economic situation and their voting behavior \parencite[e.g.,][]{Romer1975,Roberts1977,Meltzer1981}.

Although the conclusions of survey research just described have been replicated numerous times \parencite{Nie1976,Smith989,LewisBeck2008}, in the past decade these results have nonetheless faced significant challenges from scholars bringing to bear new data sources and applying new data analysis techniques. In particular, ideology and economic interest are gaining renewed recognition as determinants of life-long voting behavior. \textcite{Ansolabehere2008} show that the apparent incoherence of individual issue attitudes may largely be a result of measurement error, and that by averaging multiple question responses one gets a more stable estimate of individual issue attitudes, which become nearly as predictive of voting behavior as partisan identification.  \textcite{Gelman2007} use exit poll data to establish that income can be an important determiner of individual vote behavior, but its predictive power depends greatly on geography: in poor states like Alabama those who make little vote very differently from those who make a lot, while in rich states like Connecticut the difference is barely present. The primary innovation of both these papers is that they make compelling arguments for the use of data analysis techniques not normally used within the strand of survey research concerned with voting behavior.  Other recent papers have proposed to use entirely different data sources to explore questions about voting behavior that would have previously been answered with survey data alone. \textcite{Hersh2015}, building on the results of \textcite{Gelman2007}, use highly disaggregated registration, census, and election returns to show that income is a significant determiner of voting behavior only in Congressional districts with large minority populations. \textcite{Gerber2011} found a significant but ephemeral effect for televsion advertising by conducting a randomized experiment involving a \$2,000,000 ad-buy in a Texas gubernatorial race. \citeauthor{Arunchalam} use height as an instrument for studying the effect of income on support for the Conservative party in the United Kingdom. \textcite{Ansell2014} looks at ANES survey data from individuals living in Metropolitan Statistical Areas for which the Federal Housing Adminstration collects data on home values to calculate the effect of home value appreciation on attitudes toward a redistributive program. Ansell's substantive results, that increasing property values makes individuals less likely to support redistribution, nicely compliment ours, especially since we do not rely on any of the same data sources and uses a different identification strategy.  This Article continues in the vein of these most recent papers, using data sources and methods not frequently used in the voting behavior literature to support the claim that economic experience can have a significant effect on voting behavior. 

%textit{Works such as these  work places the common-sense notion that higher income individual typically support parties less supportive of redistribution helps to   diverse, well-identified studies establish that changes in income ca, which  work is encouraging given number of models that assume individual incentives.}


