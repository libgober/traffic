\section{Relationship to the Literature}
At least since the 1940s, social scientists have relied on survey research to understand the factors affecting voting behavior, yet only recently have scholars begun to use experimental and observational evidence to gain ``causal leverage for analyses of voting behavior'' \parencite{Bartels2010}. Early studies of voting behavior established that although individual vote choice is susceptible to election-year specific shocks, such as an unusually popular presidential candidate, the way an individual votes in most elections is for the most part a function of their personal identity characteristics \parencite{Converse1966}. According to this line of research, the most significant determinants of long-term voting behavior include party identification, ethnicity, gender, age, religion, education, and occupation \parencite{Lazarsfeld1948,Berelson1954,Campbell1960,Stanley2006a}. Noteworthy too are the factors these studies did \textit{not} find significant for long-term voting behavior: preferences about political issues, for example, and economic self-interest. This is not to say that financial well-being was ever considered irrelevant for voting behavior; the literature is replete with studies demonstrating that the success of incumbents is tied to that of the economy \parencite{Tufte1975,Meltzer1975,Hibbs1987}. Yet this effect appears to be based largely on ``sociotropic'' evaluations of national economic performance, not personal economic experience, and evidence for the latter's effect on voting behavior is at best mixed, even in the short term \parencite{Linn2010}.  For early survey research, long-term voting behavior was for the most part reducible to an individual's demographic characteristics, and would not predict that increasing an individual's wealth or income would not change their vote choice, all else equal.  This view is in considerable tension with formal models whose comparative statics expect such a relationship between an indivdual's economic situation and their voting behavior \parencite[e.g.,][]{Romer1975,Roberts1977,Meltzer1981}.

Although the conclusions of survey research just described have been replicated numerous times \parencite{Nie1976,Smith989,LewisBeck2008}, in the past decade these results have nonetheless faced significant challenges from scholars bringing to bear new data sources and applying new data analysis techniques. In particular, ideology and economic interest are gaining renewed recognition as determinants of life-long voting behavior. \textcite{Ansolabehere2008} show that the apparent incoherence of individual issue attitudes may largely be a result of measurement error, and that by averaging multiple question responses one gets a more stable estimate of individual issue attitudes, which become nearly as predictive of voting behavior as partisan identification.  \textcite{Gelman2007} use multilevel-modeling of exit poll data to establish that income can be an important determiner of individual vote behavior, but its predictive power depends greatly on geography: in poor states like Alabama those who make little vote very differently from those who make a lot, while in rich states like Connecticut the difference is barely present. \textcite{Hersh2015}, building on the results of \textcite{Gelman2007}, use highly disaggregated registration, census, and election returns to show that income is a significant determiner of voting behavior only in Congressional districts with large minority populations. \citeauthor{Arunchalam} use height as an instrument for studying the effect of income on support for the Conservative Party in the United Kingdom. 

This article continues in the vein of these most recent papers, using data sources and methods not frequently used in the voting behavior literature to support the claim that economic experience can have a significant effect on voting behavior. In particular, our decision to examine property values rather than income makes our work comparable to \textcite{Ansell2014}. In that paper, Ansell subsets ANES survey data to Metropolitain Statistical Areas where the Federal Housing Adminstration (FHA) collects data on home values. He uses these data to calculate the effect of home value appreciation on attitudes toward a redistributive program. He finds that increasing property values makes individuals less likely to support redistribution. For robustness, he looks at similar data from the UK as well as cross-country survey data. Ansell's paper nicely compliments ours. His data sources do not readily lend themselves to causal identification strategies, whereas ours do, while our outcome variables of donations and votes are only rough proxies for attitudes toward redistribution, which are better captured by surveys. Together, these papers present consistent evidence of a substantial connection that increasing property values make individuals less supportive of redistribution and more likely to support right-wing political parties. 

Our work also relies on canonical results from urban economics and economic geography.``Monocentric city models" that explain how the economy organizes space have a long history in economic thought. First proposed by \textcite{Thunen1826} to study crop usage, this model and its refinements remain widely used in the theoretical and empirical literature \parencite{Alonso,Fujita1999,BaumSnow2007}. The important implication of such models for our purposes is that the price individuals are willing to pay for residential property is inversely related to the costs of getting from that property to the city center. Our study exploits a shock to travel-costs identified by \textcite{Currie2011a}: the introduction of EZ-Pass tolling plazas. According to a study by the New Jersey Turnpike Authority, the introduction of EZ-Pass reduced total delays at toll locations 85\% in the year after its adoption, decreasing the amount of time cars were delayed by 1.8 million hours \parencite{NJT2001}. Considerable evidence supports the conclusion that residential property values are responsive to shocks that affect travel costs, and indeed we confirm that this is true for EZ-Pass introduction. Although these changes in property values should eventually result in changes to who lives in communities, in the short term individuals will not necessarily be willing or able to sell their homes. In the Appendix, we formalize these notions by developing a two period model where individuals first choose homes given a personal endowment and assuming they will have to pass a balanced budget to support the provision of public goods, and then in the second period either receive a reducation in travel costs or do not. In line with our informal predictions, the formal model produces comparative statics showing that those who receive a windfall reduction in travel costs are more reluctant to support redistributive programs.

%textit{Works such as these  work places the common-sense notion that higher income individual typically support parties less supportive of redistribution helps to   diverse, well-identified studies establish that changes in income ca, which  work is encouraging given number of models that assume individual incentives.}


