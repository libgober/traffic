The self-interested voter hypothesis suggests that support for conservative policies should increase following house price appreciation. In what follows, we show that reductions in traffic plausibly increase housing values and, in turn, conservative support. For further details on why traffic reductions should, all else equal, increase housing values, see the appendix. The basic intuition is that reductions in traffic make nearby communities more desirable, increasing the bargaining power of residents and driving up housing values. Our empirical approach to addressing this relationship exploits the fact that the introduction of E-ZPass in Pennsylvania and New Jersey resulted in decreases in traffic in some (but not all) parts of the states. In addition, E-ZPass plazas were not selected endogenously at the time of introduction, but replaced already existing toll structures.\footnote{We later perform a series of robustness checks to ensure that the initial (endogenous) placement of the tolls does not confound our inferences.} We propose IV and conditional difference-in-difference estimators to evaluate the change in political attitudes between precincts that were exposed to E-ZPass (and that thus experienced a sharp rise in property values) and those that were not (Donald and Lang 2007). 