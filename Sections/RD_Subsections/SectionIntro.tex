As our study attempts to understand how changes in individual wealth affect voting behavior, we rely on the fact that the introduction of E-ZPass served as a shock to transportation costs for some localities but not others. This shock is plausibly exogenous because E-ZPass plazas were not selected strategically at the time of introduction, but replaced already existing toll structures. We propose IV and conditional difference-in-difference estimators to evaluate the change in political attitudes between precincts that were exposed to E-ZPass (and that thus experienced a sharp rise in property values) and those that were not (Donald and Lang 2007). We also perform a series of robustness checks to ensure that the initial (endogenous) placement of the tolls does not confound our inferences.