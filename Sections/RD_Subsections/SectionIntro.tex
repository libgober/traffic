Our empirical approach to understanding how changes in individual wealth affect voting behavior relies on the fact that the introduction of E-ZPass served as a shock to transportation costs for some localities but not others. E-ZPass plazas were not selected endogenously at the time of introduction, but replaced already existing toll structures.\footnote{We later perform a series of robustness checks to ensure that the initial (endogenous) placement of the tolls does not confound our inferences.} We propose IV and conditional difference-in-difference estimators to evaluate the change in political attitudes between precincts that were exposed to E-ZPass (and that thus experienced a sharp rise in property values) and those that were not (Donald and Lang 2007). 