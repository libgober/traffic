
Next, this analysis must address a key, if subtle, study design issue. To obtain a causal estimate, it is first necessary to identify treatment and control groups, or at least some axis that is used to establish intensity of treatment and to enable researchers to compare units. However, studies that use geographic boundaries as identification mechanisms are open to criticism around how the treatment and control groups are specified. If the treatment and control groups can be defined arbitrarily, then a study may be vulnerable to criticisms over data mining. As a consequence, this analysis uses two definitions of treated and control groups in order to illustrate that our results are robust to a range of conceptual and empirical specifications. 

Our analysis first uses a conditional difference-in-differences (diff-in-diff) model with matching. This method takes as the treated group those precincts that are close to an E-ZPass exit toll plaza, which therefore are more likely to have received an exogenous increase in average home price. Whether ``close'' should mean 5, 10, or 15 miles is unclear \emph{a priori}, and thus a sensitivity analysis is required to assess the dependency of the results on how one defines ``closeness.'' We must also define a reasonable control group that could have received a reduction in traffic but did not. To construct this control group, we examine precincts close to exits on major highways without E-ZPass tolls. However, there are precincts that are both within, say, 10 miles of an E-ZPass highway and 10 miles of a highway without E-ZPass, particularly close to metropolitan areas. In order to get genuine separation of treatment and control groups, we create a rule excluding those precincts that are too close to being in the alternative testing group. The exclusion rule should have a radius at least as big as the inclusion rule to guarantee perfect separation. One should also be concerned that citizens may be willing to drive further to take a non-toll road than one with tolls. In citizens' everyday experience, perceived ``closeness'' to an E-ZPass exit may differ from perceived ``closeness'' to a non-E-ZPass exit. According to this line of reasoning, the radius of the exclusion rule should be a  larger than the radius of the inclusion rule. As the exclusion radius increases, however, the sample size necessarily decreases. The units most likely to be dropped are those closer to metropolitan areas where different highways intersect. While, in principle, treatment and control groups could each have their own inclusion and exclusion rules, we assume that treatment and control each have the same rule. We conduct our basic analysis including a precinct in a testing group if it is within 12 miles of that testing group's highway, but not within 18 miles of the alternative group's highway. We then replciate the diff-in-diff on a grid of plausible values for the exclusion and inclusion as a sensitivity test. 

Although it is reasonable to think of EZ-Pass introduction as an exogenous shock, independent of changes in Democratic vote share between the 2000 and 2004 elections, it is nonetheless still possible that treated and control precincts systematically differ in terms of relevant background covariates. To address this possibility, which is present even in randomized studies, we match on key covariates which might confound our estimates \parencite{Morgan2012}. Matching also reduces model sensitivity \parencite{Ho2007}, narrowing the range of estimated effects. In the matching routine, we use propensity score matching with a caliper of $0.20$. Matching was done without replacement. We also attempted Mahalanobis distance matching and coarsened exact matching, but these approaches gave poorer balance.

Our analysis next uses an instrumental variables (IV) approach. This method does not posit the existence of a separate treated and control group. Rather, we use distance to the nearest E-ZPass toll plaza as an instrument for home price appreciation.\footnote{For the purposes of this IV analysis, we examine only those precincts within $20$ miles of E-ZPass plazas.} This framework depends on two assumptions. First, distance to E-ZPass must be correlated with the endogenous explanatory variable (i.e. gain in home price). This correlation is strong, with a $t$-statistic of over $10$. Next, the instrument cannot be correlated with the error term of our explanatory equation predicting change in Democratic $2$-party vote share. This assumption would be violated if distance to E-ZPass toll plazas affected $2$-party vote share even when home prices are kept constant. 

