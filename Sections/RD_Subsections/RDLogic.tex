This analysis must address a key issue around internal validity. That is, to obtain any causal estimate, it is first necessary to identify treatment and control groups. However, studies that use geographic boundaries as identification mechanisms are open to criticism around how the treatment and control groups are specified. If the treatment and control groups can be defined arbitrarily, then a study may be vulnerable to data mining problems. As a consequence, this analysis uses two definitions of treated and control groups in order to illustrate that our results are robust to these two specifications. 

Our analysis first uses a conditional difference-in-differences model with matching. This method takes as the treated group those precincts living corollary close to an E-ZPass exit toll plaza, who therefore are more likely to have received an exogenous increase in their housing price. Whether ``close'' should mean 5, 10, or 15 miles is unclear \emph{a priori}, thus a sensitivity analysis is required to assess the dependency of the results on how one defines ``closeness."" We must also define a reasonable control group that could have received a reduction in traffic but did not. To construct this control group, we examine precincts close to exits on major highways without E-ZPass tolls. However, particularly close to metropolitan areas, there are precincts that are both within 10 miles of an E-ZPass highway and 10 miles of a highway without E-ZPass.  In order to get genuine separation of treatment and control groups, we create a rule excluding those precincts that are in one testing group but are on the cusp of being on the other. The exclusion rule should have a radius at least as big as the inclusion rule to guarantee perfect separation. But one should also be concerned that citizens may be willing to drive further to take a non-toll road than one with tolls (i.e. in citizens' everyday experience, perceived ``closeness'' to an E-ZPass exit may differ from perceived ``closeness'' to a non-E-ZPass exit). According to this line of reasoning, the radius of the exclusion rule should be a bit larger than the radius of the inclusion rule. If the exclusion radius is too large, however, one will exclude many units in metropolitan areas where different highways intersect. While, in principle, treatment and control groups could each have their own inclusion and exclusion rules, we shall assume that treatment and control each have the same rule for inclusion and the same rule for exclusion.  We conduct our basic analysis including a precinct in a testing group if it is within 12 miles of the highway it belongs in, but not within 18 miles of a highway it should not belong in. We then replciate the diff-in-diff on a grid of plausible values.

Our analysis next uses an instrumental variables (IV) approach. This method does not posit the existence of a separate treated and control group. Rather, we use distance to an E-ZPass toll plaza as an instrument for housing price appreciation for those precincts within some fixed distance from the toll plaza. This framework depends on two assumptions. First, distance to E-ZPass toll plaza must be correlated with the endogenous explanatory variable (i.e. gain in housing price). This correlation is strong (with a $t$-statistic of over $20$). Next, the instrument cannot be correlated with the error term of our explanatory equation predicting change in Democratic $2$-party vote share. This assumption would be violated if distance to E-ZPass toll plazas affected Democratic $2$-party vote share even when housing prices are kept constant. 

In what follows, we show that our results are robust to the specification of the treated and control groups, as well as to the method of inference. We have a slight preference for the conditional difference-in-difference approach because this specification of treated and control groups seems more natural. Areas $2$ vs. $10$ miles from an E-ZPass toll plaza may be more systematically different from each other than precincts near E-ZPass toll plazas and those near non-E-ZPass traffic exits. Even so, it is important to give careful thought to the specification of the treated and control groups in geographically-based analyses, and it is best to show the robustness of a result across a range of specifications.  We also take care to show that other important predictors of Democratic support do not undergo significant changes following the introduction of E-ZPass. 