For this study, we rely on a combination of low-level voting, political contribution, census and geographic data. All geographic analysis, in particular data about the relative distance of two locations, were conducted using ArcGIS with a national highway map provided by the developer of ArcGIS. The location of EZ-Pass tolling booths were taken from the replication dataset to \textcite{Currie2011a}, and this data was replicated and supplemented with data collected from Department of Transportation websites. Precinct-level data on the number of registered voters and the number of votes received by each party in Presidential elections, as well as shape files detailing the geography location of each precinct, were taken from \textcite{Ansolabehere2014}. Contribution data collected by the Federal Election Commission (FEC) was reported at the individual level, however this data is reported by zipcode. Census data used for matching and for constructing our instrumental variable were taken from the 2000 decenial census and the 2005-2009 American Community Survey (ACS). 

Because these data are reported at different levels of aggregation, substantial effort was required to create a dataset suitable for analysis. Formally, we consider an observational unit in our study to be the voting precinct. Voting precincts are contiguous areas, typically about 5 square miles in area, and are roughly the same size as a census block group, the smallest area at which census data are reported. Zipcodes are usually much larger areas containing multiple precincts or census block groups. Since the boundaries of precincts, census block groups, and zipcodes are generally not identical, we use areal interpolation to impute the data from these other geographic area to the precincts. ArcPython replication code is provided for those interested in seeing exactly how each column in our dataset was constructed, however the basic idea is no more complicated than taking a weighted average, with weights based on the amount of area that overlaps between the precinct and the other geographic area one is interpolating. Also crucial to our analysis is the distance of each precinct to the highway. For this purpose we consider the polling place as coded in \textcite{Ansolabehere2014} to be the precinct's location, and we operationalize distance to the highway to be the network or ``over-road distance" to the nearest highway entrance.  