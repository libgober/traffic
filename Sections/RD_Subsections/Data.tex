\\ 
\indent We rely on a combination of low-level voting, political contribution, census and geographic data. All geographic analysis were conducted using ArcGIS with a national highway map provided by the developer of ArcGIS. The location of EZ-Pass tolling booths were taken from the replication dataset to \textcite{Currie2011a}, and this data was replicated and supplemented with data collected from Department of Transportation websites. 

We use a variety of measures for our outcome variable of Democratic support, which we proxy in three ways. First, we examine the two-party Democratic vote share in the presidential elections of 2000, 2004, and 2008. This quantity is defined as the total number of votes for the Democratic candidate in each election divided by the total number of votes cast for the Democratic and Republican candidates. Next, we also examine two-party Democratic cash share raised in the 2000 and 2004 election cycles. This quantity is defined as the total dollar amount of campaign contributions given to Democratic candidates in these elections divided by the total dollar amount contributed to either Republican or Democratic candidates. Finally, we also examine the support from stationary campaign contributors. Stationary contributors are defined as citizens residing at the same address in 2000 and 2004, and who contributed to presidential candidates in these years. Here, we can proxy Democratic support at the precinct level by taking the number of support of the Democratic candidate divided by the total number of stationary contributors in each precinct. Precinct-level data on the number of registered voters and the number of votes received by each party in Presidential elections, as well as shape files detailing the geography location of each precinct, were taken from \textcite{Ansolabehere2014}. Contribution data collected by the Federal Election Commission (FEC) was reported at the individual level by zipcode, and address information for contributors is available. 

Census data used for matching and for obtaining home price statistics were taken from the 2000 decenial census and the 2005-2009 American Community Survey (ACS). Our matching variables are precinct-level statistics, and include average income, percentage of the population with a bachelors or professional degree, percentage of the population which identifies as black, percentage of the population which is female, percentage of the population which is over the age of 65, and percentage of the population residing in the same house as in 95\%. Previous literature has found that these variables serve as importnat predictors of voting behavior in presidential elections (CITATIONS NEEDED). 

Because these data are reported at different levels of aggregation, substantial effort was required to create a dataset suitable for analysis. Formally, we consider an observational unit in our study to be the voting precinct. Voting precincts are contiguous areas, typically about 5 square miles in area, and are roughly the same size as a census block group, the smallest area at which census data are reported. Zipcodes are usually much larger areas containing multiple precincts or census block groups. Since the boundaries of precincts, census block groups, and zipcodes are generally not identical, we use areal interpolation to impute the data from these other geographic area to the precincts. ArcPython replication code is provided for those interested in seeing exactly how each column in our dataset was constructed. The intuition is no more complicated than taking a weighted average, with weights based on the amount of area that overlaps between the precinct and the other geographic area one is interpolating. Also crucial to our analysis is the distance of each precinct to the highway. For this purpose we consider the polling place as coded in \textcite{Ansolabehere2014} to be the precinct's location, and we operationalize distance to the highway (or an E-ZPass plaza) to be the minimum network or ``over-road distance" to the nearest highway entrance (or to the E-ZPass plaza).  