
Any diff-in-diff analysis is also threatened by the presence of time-varying confounders. The effect of the E-ZPass intervention might be correlated with other changes occurring in treated or control communities. Although we have little leverage in directly identifying unobserved confounding, we can establish empirically whether key observed confounders might be changing along with our outcomes of interest. In particular, we can identify such effects by completing the same diff-in-diff procedure for other factors that are important for predicting Democratic vote share. We find that no key predictors of Democratic vote share underwent changes associated with the introduction of the E-ZPass. 

Table \ref{alt_explanations} shows that average incomes seem to have remained stable in treated E-ZPass precincts relative to their control counterparts. The diff-in-diff estimate ranges from -$\$613$ to -$\$1,450$, but are statistically insignificant. We might have been concerned that E-ZPass might have pushed out poor residents and led to an influx of richer ones. However, this worry seems unfounded. Indeed, the sign suggests that incomes fell in E-ZPass precincts relative to control areas. Next, since E-ZPass precincts are now more desirable, it could be that these communities experienced an influx of new, potentially more conservative residents. However, we find that, on average, E-ZPass communities did not see a population change compared to the control areas. Changes in turnout might be another confounding variable. If E-ZPass significantly increase or decreased turnout, inferences about the changes in the average Democratic vote share might be suspect. Here too, we find no effect of E-ZPass. Finally, we might wonder whether other characteristics of the E-ZPass communities were affected by the introduction of the electronic tolls. Could it be that E-ZPass communities experienced changes in racial or educational composition? No, neither the percentage of black residents nor the percentage of residents with a bachelor's degree changed in treated relative to control precincts. 

\begin{table}[!tbp] \centering 
\caption{Diff-in-diff results for other key variables.}
\resizebox{0.95\textwidth}{!}{
\begin{tabular}{@{\extracolsep{5pt}} ccccc} 
\\[-1.8ex]\hline 
\hline \\[-1.8ex] 
& \multicolumn{2}{c}{\emph{Baseline model}} & \multicolumn{2}{c}{\emph{With covariate adjustment}} \\
& \multicolumn{2}{c}{\textemdash} & \multicolumn{2}{c}{\textemdash} \\
\hline \\[-1.8ex] 
Dependent variable & DiD Estimate & (Block Bootstrap S.D.) & DiD Estimate & (Block Bootstrap S.D.) \\ 
\hline \\[-1.8ex] 
\emph{Average income} & -613.81 & (2488.12) & -1830.39 & (2572.56) \\ 
\emph{Population} & 30.88 & (30.17) & 15.27 & (32.61) \\ 
\emph{Turnout} & 0.72 & (0.79) & 0.61 & (0.82) \\ 
\emph{Percent with bachelors} & 0.64 & (0.37) & 0.56 & (0.43) \\ 
\emph{Percent black} & 0.47 & (0.35) & 0.59 & (0.33) \\ 
\hline \\[-1.8ex] 
\end{tabular} 
} 
 \label{alt_explanations} 
\end{table} 

